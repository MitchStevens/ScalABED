\documentclass[a4paper, 12pt]{article}

\begin{document}
\tableofcontents
	
\section{Tutorial}
\subsection{Introduction to BabyTown}
\begin{tabular}{ccc}
	Min Size & Accepts & Returns \\ 
	3 & (Left 1) & (Right 1)
\end{tabular}

\subparagraph{Instruction Text}
Place an input on the left and an output on the right. Then connect them by using bus pieces to extend the range of the input signal.

\subparagraph{Tests}

\begin{tabular}{ c c }
	Input & Output \\ 
	0 & 0 \\  
	1 & 1     
\end{tabular}

\subparagraph{Completion Text}
You did it! The easiest level in the game! Hopefully the game gets harder than this (Right?). Those pieces we were playing with a moment ago are called ‘Gates’, because they stop or allow the flow of a signal. This signal is represented by an electric blue current in ABED.

\subparagraph{Upon Completion}
Unlock circuit 'NOT', Unlock level '1.2'

\subsection{Is Nice?}
\begin{tabular}{ccc}
	Min Size & Accepts & Returns \\ 
	3 & (Left 1) & (Right 1)
\end{tabular}

\subparagraph{Instruction Text}
Having completed 'Introduction to Babytown', you should now have a a new gate called a 'Not' gate. Now we are going to test out our new gate. Place an input on the left and an output on the right as before. This time, create a Not gate in the center instead of a Bus.

\subparagraph{Tests}
\begin{tabular}{ c c }
	Input & Output \\ 
	0 & 1 \\  
	1 & 0     
\end{tabular}

\subparagraph{Completion Text}
Another success! A Not gate negates the signal coming from the input. So if the input of a Not gate is off, it will output on, and vice versa. In general, the input of a gate is on the left and the output is on the right. But as we will see, we are by no means restricted to one input/output.

\subsection{My First Three-way!}
\begin{tabular}{ccc}
	Min Size & Accepts & Returns \\ 
	3 & (Left 1) (Up 1) & (Right 1)
\end{tabular}

\subparagraph{Instruction Text}
Oh boy, two inputs? Create an Or gate and place inputs at 3, 0 and an output at 1. Remember to rotate the input so the signal is facing the Or gate.

\subparagraph{Tests}
\begin{tabular}{cc}
	Input & Output  \\
	00 & 0 \\
	01 & 1 \\
	10 & 1 \\
	11 & 1
\end{tabular}

\subparagraph{Completion Text}		
Now we're getting somewhere! An 'Or' gate will output a signal if one or more of its inputs is turned on, otherwise it outputs nothing. Imagine of a waiter asking “Would you like milk OR sugar in your coffee?” Gates can have as many inputs/outputs as they damn well please, with one condition: there must at least one. Obviously.

\subsection{Circuits on Circuits?}

\subsection{Andromeda (A whole new galaxy!)}

\subsection{Mr. NAND-Man}

\section{Basic Gates}

\subsection{Make a Left}
\subsection{XOR Problems}
\subsection{What is this, a crossover episode?}
\subsection{Scrublords Delight}

\section{Advanced Gates}

\subsection{A Two Parter}
A double bus
\subsection{It's French for Ship!}
Using a merge, recreate a merge
\subsection{The Opposite of Merge!}
using a branch, recreate a branch
\subsection{The ol' Digital Sage Switcharoo}
Using a merge, a branch, and a crossover, reverse the input so that the first element in the signal is second and vice versa
\subsection{Double Negative}
Using a merge, brnach and a two nots, negate the input.
\subsection{Fine and Double-And-y}
simulate a bitwise and gate
\subsection{OK, we get it!}
Create a 4 part bus.

\section{Mux and Demux}

\subsection{Multiplexer}
\subsection{Megaplexer}
\subsection{Demultiplexer}
\subsection{}

\section{Binary}
\subsection{Neo is One Backwards!}
\begin{tabular}{ccc}
	Min Size & Accepts & Returns \\ 
	4 & None & (Right 4)
\end{tabular}

\subparagraph{Instruction Text}
Now that you have a Display, we can start to learn about binary! 

\subparagraph{Tests}
\begin{tabular}{cc}
	Input & Output \\ 
	None & 0001
\end{tabular}

\subparagraph{Completion Text}

\section{Number Fun}
\begin{tabular}{ccc}
	Min Size & Accepts & Returns \\
	4 & None & (Right 4)
\end{tabular}

\subparagraph{Instruction Text}

\subparagraph{Tests}


\subparagraph{Completion Text}

\end{document}

Example document

\section{Name}
\begin{tabular}{ccc}
	Min Size & Accepts & Returns \\
	
\end{tabular}

\subparagraph{Instruction Text}

\subparagraph{Tests}
\begin{tabular}{cc}
	Input & Output \\
\end{tabular}

\subparagraphs{Completion Text}