\documentclass[a4paper, 12pt]{article}

\begin{document}
\tableofcontents
%%%%%%%%%%%%%%%%%%%%%%%%%%%%%%%%%%%%%%%%%%%%%%%%%%%%%%%%%%%%%%%%%%%%%%%%%%
\section{Tutorial}
\subsection{Introduction to BabyTown}
\begin{tabular}{ccc}
	Min Size & Accepts & Returns \\ 
	3 & (Left 1) & (Right 1)
\end{tabular}

\subparagraph{Instruction Text}
Place an input on the left and an output on the right. Then connect them by using bus pieces to extend the range of the input signal.

\subparagraph{Tests}

\begin{tabular}{ c c }
	Input & Output \\ 
	0 & 0 \\  
	1 & 1     
\end{tabular}

\subparagraph{Completion Text}
You did it! The easiest level in the game! Hopefully the game gets harder than this (Right?). Those pieces we were playing with a moment ago are called ‘Gates’, because they stop or allow the flow of a signal. This signal is represented by an electric blue current in ABED.

\subparagraph{Upon Completion}
Unlock circuit 'NOT', Unlock level '1.2'
%%%%%%%%%%%%%%%%%%%%%%%%%%%%%%%%%%%%%%%%%%%%%%%%%%%%%%%%%%%%%%%%%%%%%%%%%%
\subsection{Is Nice?}
\begin{tabular}{ccc}
	Min Size & Accepts & Returns \\ 
	3 & (Left 1) & (Right 1)
\end{tabular}

\subparagraph{Instruction Text}
Having completed 'Introduction to Babytown', you should now have a a new gate called a 'Not' gate. Now we are going to test out our new gate. Place an input on the left and an output on the right as before. This time, create a Not gate in the center instead of a Bus.

\subparagraph{Tests}
\begin{tabular}{ c c }
	Input & Output \\ 
	0 & 1 \\  
	1 & 0     
\end{tabular}

\subparagraph{Completion Text}
Another success! A Not gate negates the signal coming from the input. So if the input of a Not gate is off, it will output on, and vice versa. In general, the input of a gate is on the left and the output is on the right. But as we will see, we are by no means restricted to one input/output.
%%%%%%%%%%%%%%%%%%%%%%%%%%%%%%%%%%%%%%%%%%%%%%%%%%%%%%%%%%%%%%%%%%%%%%%%%%
\subsection{My First Three-way!}
\begin{tabular}{ccc}
	Min Size & Accepts & Returns \\ 
	3 & (Left 1) (Up 1) & (Right 1)
\end{tabular}

\subparagraph{Instruction Text}
Oh boy, two inputs? Create an Or gate and place inputs at 3, 0 and an output at 1. Remember to rotate the input so the signal is facing the Or gate.

\subparagraph{Tests}
\begin{tabular}{cc}
	Input & Output  \\
	00 & 0 \\
	01 & 1 \\
	10 & 1 \\
	11 & 1
\end{tabular}

\subparagraph{Completion Text}		
Now we're getting somewhere! An 'Or' gate will output a signal if one or more of its inputs is turned on, otherwise it outputs nothing. Imagine of a waiter asking “Would you like milk OR sugar in your coffee?” Gates can have as many inputs/outputs as they damn well please, with one condition: there must at least one. Obviously.
%%%%%%%%%%%%%%%%%%%%%%%%%%%%%%%%%%%%%%%%%%%%%%%%%%%%%%%%%%%%%%%%%%%%%%%%%%
\subsection{Circuits on Circuits?}
\subsection{Name}
\begin{tabular}{ccc}
	Min Size & Accepts & Returns \\
	4 & (Left 1) (Up 1) & (Right 1)
\end{tabular}

\subparagraph{Instruction Text}
As it turns out, we can also create completely original circuits from those we already have. Create another Or, but this time use a not gate to negate the output. You may need to change the size of the circuit board.

\subparagraph{Tests}
\begin{tabular}{cc}
	Input & Output \\
	00 & 1 \\
	01 & 0 \\
	10 & 0 \\
	11 & 0
\end{tabular}

\subparagraph{Completion Text}
And there you go, a whole new gate. Once you complete a level, the game is turned into a circuit for you to use on any game you please. You can even create circuits without completing a level on sandbox mode.

%%%%%%%%%%%%%%%%%%%%%%%%%%%%%%%%%%%%%%%%%%%%%%%%%%%%%%%%%%%%%%%%%%%%%%%%%%
\subsection{Andromeda (A whole new galaxy!)}
\begin{tabular}{ccc}
	Min Size & Accepts & Returns \\
	5 & (Left 1) (Up 1) & (Right 1)
\end{tabular}

\subparagraph{Instruction Text}
Create an And gate. This gate takes two inputs at 3, 0 and outputs a signal if both of the inputs are on. If one or more of the outputs are off, the output should also be off.

\subparagraph{Tests}
\begin{tabular}{cc}
	Input & Output \\
	00 & 0 \\
	01 & 0 \\
	10 & 0 \\
	11 & 1
\end{tabular}

\subparagraph{Completion Text}
Hey you did it! I was worried I'd lose you there for a second. And, Not and Or are the fundamental operations of Boolean algebra, every electronic circuit in your computer is made exclusively of these gates. Now you’ve created these three, we can begin to really make things!

%%%%%%%%%%%%%%%%%%%%%%%%%%%%%%%%%%%%%%%%%%%%%%%%%%%%%%%%%%%%%%%%%%%%%%%%%%
\subsection{Mr. NAND-Man}
\begin{tabular}{ccc}
	Min Size & Accepts & Returns \\
	4 & (Left 1) (Up 1) & (Right 1)
\end{tabular}

\subparagraph{Instruction Text}
Negate an And gate.

\subparagraph{Tests}
\begin{tabular}{cc}
	Input & Output \\
	00 & 1 \\
	01 & 1 \\
	10 & 1 \\
	11 & 0
\end{tabular}

\subparagraph{Completion Text}
Yeah it was pretty easy, but you got a new circuit out of it. And you gotta collect ‘em all! You feel like Ash Ketchum yet, you piece of millennial trash?

%%%%%%%%%%%%%%%%%%%%%%%%%%%%%%%%%%%%%%%%%%%%%%%%%%%%%%%%%%%%%%%%%%%%%%%%%%
\section{Basic Gates}

\subsection{Make a Left}
\begin{tabular}{ccc}
	Min Size & Accepts & Returns \\
	3 & (Left 1) & (Up 1)
\end{tabular}

\subparagraph{Instruction Text}
Create a circuit that takes and input on 3 and outputs the signal to an output on side 0.

\subparagraph{Tests}
\begin{tabular}{cc}
	Input & Output \\
	0 & 0 \\
	1 & 1
\end{tabular}

\subparagraph{Completion Text}
Wow, writing these is really tiresome. These gates should give you more freedom in choosing where to place tiles.
%%%%%%%%%%%%%%%%%%%%%%%%%%%%%%%%%%%%%%%%%%%%%%%%%%%%%%%%%%%%%%%%%%%%%%%%%%
\subsection{XOR Problems}
\begin{tabular}{ccc}
	Min Size & Accepts & Returns \\
	7 & (Left 1) (Up 1) & (Right 1)
\end{tabular}

\subparagraph{Instruction Text}
This one’s tricky. We now need to create an 'XOR' gate. The name XOR comes from the phrase 'exclusive or', which means either of the two, but not both.

Create a gate that outputs a signal if exactly one input is on. If both inputs are on or both inputs are off, then output off. The easiest way o do this is by using 4 NAND gates, 
\subparagraph{Tests}
\begin{tabular}{cc}
	Input & Output \\
	00 & 0 \\
	01 & 1 \\
	10 & 1 \\
	11 & 0
\end{tabular}

\subparagraph{Completion Text}
Oh boy, this one was a good one. Did you have to use google? No shame if you did.

%%%%%%%%%%%%%%%%%%%%%%%%%%%%%%%%%%%%%%%%%%%%%%%%%%%%%%%%%%%%%%%%%%%%%%%%%%
\subsection{What is this, a crossover episode?}
\begin{tabular}{ccc}
	Min Size & Accepts & Returns \\
	
\end{tabular}

\subparagraph{Instruction Text}

\subparagraph{Tests}
\begin{tabular}{cc}
	Input & Output \\
	00 & 00 \\
	01 & 10 \\
	10 & 01 \\
	11 & 11
\end{tabular}

\subparagraph{Completion Text}
%%%%%%%%%%%%%%%%%%%%%%%%%%%%%%%%%%%%%%%%%%%%%%%%%%%%%%%%%%%%%%%%%%%%%%%%%%
\subsection{Scrublords Delight}
\begin{tabular}{ccc}
	Min Size & Accepts & Returns \\
	
\end{tabular}

\subparagraph{Instruction Text}

\subparagraph{Tests}
\begin{tabular}{cc}
	Input & Output \\
\end{tabular}

\subparagraph{Completion Text}
%%%%%%%%%%%%%%%%%%%%%%%%%%%%%%%%%%%%%%%%%%%%%%%%%%%%%%%%%%%%%%%%%%%%%%%%%%
\section{Advanced Gates}
\begin{tabular}{ccc}
	Min Size & Accepts & Returns \\
	
\end{tabular}

\subparagraph{Instruction Text}

\subparagraph{Tests}
\begin{tabular}{cc}
	Input & Output \\
\end{tabular}

\subparagraph{Completion Text}
%%%%%%%%%%%%%%%%%%%%%%%%%%%%%%%%%%%%%%%%%%%%%%%%%%%%%%%%%%%%%%%%%%%%%%%%%%

\subsection{A Two Parter}
A double bus
%%%%%%%%%%%%%%%%%%%%%%%%%%%%%%%%%%%%%%%%%%%%%%%%%%%%%%%%%%%%%%%%%%%%%%%%%%
\subsection{It's French for Ship!}
Using a merge, recreate a merge
%%%%%%%%%%%%%%%%%%%%%%%%%%%%%%%%%%%%%%%%%%%%%%%%%%%%%%%%%%%%%%%%%%%%%%%%%%
\subsection{The Opposite of Merge!}
using a branch, recreate a branch
%%%%%%%%%%%%%%%%%%%%%%%%%%%%%%%%%%%%%%%%%%%%%%%%%%%%%%%%%%%%%%%%%%%%%%%%%%
\subsection{The ol' Digital Sage Switcharoo}
Using a merge, a branch, and a crossover, reverse the input so that the first element in the signal is second and vice versa
%%%%%%%%%%%%%%%%%%%%%%%%%%%%%%%%%%%%%%%%%%%%%%%%%%%%%%%%%%%%%%%%%%%%%%%%%%
\subsection{Double Negative}
Using a merge, brnach and a two nots, negate the input.
%%%%%%%%%%%%%%%%%%%%%%%%%%%%%%%%%%%%%%%%%%%%%%%%%%%%%%%%%%%%%%%%%%%%%%%%%%
\subsection{Fine and Double-And-y}
simulate a bitwise and gate
%%%%%%%%%%%%%%%%%%%%%%%%%%%%%%%%%%%%%%%%%%%%%%%%%%%%%%%%%%%%%%%%%%%%%%%%%%
\subsection{OK, we get it!}
Create a 4 part bus.
%%%%%%%%%%%%%%%%%%%%%%%%%%%%%%%%%%%%%%%%%%%%%%%%%%%%%%%%%%%%%%%%%%%%%%%%%%
\section{Mux and Demux}

\subsection{Multiplexer}
%%%%%%%%%%%%%%%%%%%%%%%%%%%%%%%%%%%%%%%%%%%%%%%%%%%%%%%%%%%%%%%%%%%%%%%%%%
\subsection{Demultiplexer}
%%%%%%%%%%%%%%%%%%%%%%%%%%%%%%%%%%%%%%%%%%%%%%%%%%%%%%%%%%%%%%%%%%%%%%%%%%
\subsection{Advanced BabyTown}
%%%%%%%%%%%%%%%%%%%%%%%%%%%%%%%%%%%%%%%%%%%%%%%%%%%%%%%%%%%%%%%%%%%%%%%%%%
\subsection{}
%%%%%%%%%%%%%%%%%%%%%%%%%%%%%%%%%%%%%%%%%%%%%%%%%%%%%%%%%%%%%%%%%%%%%%%%%%
\section{Binary}
\subsection{Neo is One Backwards!}
\begin{tabular}{ccc}
	Min Size & Accepts & Returns \\ 
	4 & None & (Right 4)
\end{tabular}

\subparagraph{Instruction Text}
Now that you have a Display, we can start to learn about binary! 

\subparagraph{Tests}
\begin{tabular}{cc}
	Input & Output \\ 
	None & 0001
\end{tabular}

\subparagraph{Completion Text}
%%%%%%%%%%%%%%%%%%%%%%%%%%%%%%%%%%%%%%%%%%%%%%%%%%%%%%%%%%%%%%%%%%%%%%%%%%
\subsection{Number Fun}
\begin{tabular}{ccc}
	Min Size & Accepts & Returns \\
	4 & None & (Right 4)
\end{tabular}

\subparagraph{Instruction Text}

\subparagraph{Tests}


\subparagraph{Completion Text}
%%%%%%%%%%%%%%%%%%%%%%%%%%%%%%%%%%%%%%%%%%%%%%%%%%%%%%%%%%%%%%%%%%%%%%%%%%
\section{Bitwise Operations}
\subsection{Right Shift}
%%%%%%%%%%%%%%%%%%%%%%%%%%%%%%%%%%%%%%%%%%%%%%%%%%%%%%%%%%%%%%%%%%%%%%%%%%
\subsection{Left Shift}
%%%%%%%%%%%%%%%%%%%%%%%%%%%%%%%%%%%%%%%%%%%%%%%%%%%%%%%%%%%%%%%%%%%%%%%%%%
\subsection{Bitwise Not}
%%%%%%%%%%%%%%%%%%%%%%%%%%%%%%%%%%%%%%%%%%%%%%%%%%%%%%%%%%%%%%%%%%%%%%%%%%
\subsection{Bitwise OR}
%%%%%%%%%%%%%%%%%%%%%%%%%%%%%%%%%%%%%%%%%%%%%%%%%%%%%%%%%%%%%%%%%%%%%%%%%%
\subsection{Bitwise AND}
%%%%%%%%%%%%%%%%%%%%%%%%%%%%%%%%%%%%%%%%%%%%%%%%%%%%%%%%%%%%%%%%%%%%%%%%%%
\subsection{Bitwise XOR}
%%%%%%%%%%%%%%%%%%%%%%%%%%%%%%%%%%%%%%%%%%%%%%%%%%%%%%%%%%%%%%%%%%%%%%%%%%
\section{Arithmetic Operation}
\subsection{Half Adder}
%%%%%%%%%%%%%%%%%%%%%%%%%%%%%%%%%%%%%%%%%%%%%%%%%%%%%%%%%%%%%%%%%%%%%%%%%%
\subsection{Full Adder}
%%%%%%%%%%%%%%%%%%%%%%%%%%%%%%%%%%%%%%%%%%%%%%%%%%%%%%%%%%%%%%%%%%%%%%%%%%
\subsection{2 bit adder}
%%%%%%%%%%%%%%%%%%%%%%%%%%%%%%%%%%%%%%%%%%%%%%%%%%%%%%%%%%%%%%%%%%%%%%%%%%
\subsection{Adder}
%%%%%%%%%%%%%%%%%%%%%%%%%%%%%%%%%%%%%%%%%%%%%%%%%%%%%%%%%%%%%%%%%%%%%%%%%%
\subsection{Subtract}
%%%%%%%%%%%%%%%%%%%%%%%%%%%%%%%%%%%%%%%%%%%%%%%%%%%%%%%%%%%%%%%%%%%%%%%%%%
\subsection{Multiplication}
%%%%%%%%%%%%%%%%%%%%%%%%%%%%%%%%%%%%%%%%%%%%%%%%%%%%%%%%%%%%%%%%%%%%%%%%%%
\subsection{Divide}
%%%%%%%%%%%%%%%%%%%%%%%%%%%%%%%%%%%%%%%%%%%%%%%%%%%%%%%%%%%%%%%%%%%%%%%%%%
\subsection{Modulus}
%%%%%%%%%%%%%%%%%%%%%%%%%%%%%%%%%%%%%%%%%%%%%%%%%%%%%%%%%%%%%%%%%%%%%%%%%%

\section{Memory}

\subsection{RS Latch}
%%%%%%%%%%%%%%%%%%%%%%%%%%%%%%%%%%%%%%%%%%%%%%%%%%%%%%%%%%%%%%%%%%%%%%%%%%
\subsection{1 Bit of memory}
%%%%%%%%%%%%%%%%%%%%%%%%%%%%%%%%%%%%%%%%%%%%%%%%%%%%%%%%%%%%%%%%%%%%%%%%%%
\subsection{2 Bits!}
%%%%%%%%%%%%%%%%%%%%%%%%%%%%%%%%%%%%%%%%%%%%%%%%%%%%%%%%%%%%%%%%%%%%%%%%%%
\subsection{A byte of memory}
%%%%%%%%%%%%%%%%%%%%%%%%%%%%%%%%%%%%%%%%%%%%%%%%%%%%%%%%%%%%%%%%%%%%%%%%%%

\section{RAM}
%%%%%%%%%%%%%%%%%%%%%%%%%%%%%%%%%%%%%%%%%%%%%%%%%%%%%%%%%%%%%%%%%%%%%%%%%%

\section{CPU}

%%%%%%%%%%%%%%%%%%%%%%%%%%%%%%%%%%%%%%%%%%%%%%%%%%%%%%%%%%%%%%%%%%%%%%%%%%


\end{document}
%%%%%%%%%%%%%%%%%%%%%%%%%%%%%%%%%%%%%%%%%%%%%%%%%%%%%%%%
Example document

\subsection{Name}
\begin{tabular}{ccc}
	Min Size & Accepts & Returns \\
	
\end{tabular}

\subparagraph{Instruction Text}

\subparagraph{Tests}
\begin{tabular}{cc}
	Input & Output \\
\end{tabular}

\subparagraph{Completion Text}